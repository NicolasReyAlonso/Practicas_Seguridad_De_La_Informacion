\chapter{Cifrado Simétrico de Documentos}

\section{Descripción teórica}

El cifrado simétrico es una técnica criptográfica en la que se utiliza la misma clave tanto para cifrar como para descifrar. Los elementos principales son:

\begin{itemize}[label=\textbullet]
    \item \textbf{Texto claro:} Documento original sin cifrar
    \item \textbf{Algoritmo de cifrado:} Función matemática que transforma el texto
    \item \textbf{Clave:} Información secreta que controla la transformación
    \item \textbf{Texto cifrado:} Resultado ilegible del proceso
    \item \textbf{Vector de inicialización (IV):} Valor aleatorio para ciertos modos
\end{itemize}

\section{Algoritmos de cifrado simétrico}

\subsection{Cifradores de bloque}

\begin{table}[H]
    \centering
    \caption{Cifradores de bloque empleados}
    \begin{tabularx}{\textwidth}{l|c|c|p{5cm}}
        \toprule
        \textbf{Algoritmo} & \textbf{Tamaño bloque} & \textbf{Tamaño clave} & \textbf{Estado} \\
        \midrule
        AES-256 & 128 bits & 256 bits & Recomendado \\
        3DES/TDES & 64 bits & 168 bits & Deprecado \\
        \bottomrule
    \end{tabularx}
\end{table}

\subsection{Cifradores de flujo}

\begin{table}[H]
    \centering
    \caption{Cifradores de flujo empleados}
    \begin{tabularx}{\textwidth}{l|c|p{5cm}}
        \toprule
        \textbf{Algoritmo} & \textbf{Tamaño clave} & \textbf{Observaciones} \\
        \midrule
        RC4 & Variable & Débil, deprecado \\
        ChaCha20 & 256 bits & Moderno, recomendado \\
        \bottomrule
    \end{tabularx}
\end{table}

\section{Modos de operación}

\begin{table}[H]
    \centering
    \caption{Modos de operación de cifrado en bloque}
    \begin{tabularx}{\textwidth}{l|p{6cm}|p{3cm}}
        \toprule
        \textbf{Modo} & \textbf{Descripción} & \textbf{Seguridad} \\
        \midrule
        ECB & Cada bloque cifra de forma independiente & \textcolor{red}{Inseguro} \\
        CBC & Encadenamiento de bloques con IV & Seguro \\
        OFB & Flujo de retroalimentación de salida & Seguro \\
        CTR & Modo contador & Seguro \\
        GCM & Authenticated encryption & Muy seguro \\
        \bottomrule
    \end{tabularx}
\end{table}

\section{Derivación de claves: PBKDF2}

El estándar PKCS\#5 define los siguientes métodos de derivación:

\begin{itemize}[label=\textbullet]
    \item \textbf{PBKDF1:} Primera versión, limitada
    \item \textbf{PBKDF2:} Versión mejorada, recomendada
\end{itemize}

PBKDF2 genera una clave a partir de una contraseña utilizando:

\begin{equation}
DK = PBKDF2(P, S, c, dkLen)
\end{equation}

Donde:
\begin{itemize}
    \item $P$: Contraseña
    \item $S$: Sal (salt) - mínimo 8 bytes
    \item $c$: Número de iteraciones (por defecto 10000)
    \item $dkLen$: Longitud deseada de la clave
\end{itemize}

\section{Metodología experimental}

\subsection{Archivo de prueba}

Se utiliza un archivo de texto pequeño (31-81 caracteres):

\begin{lstlisting}[caption=Archivo para cifrado simétrico]
Este es un texto de prueba para cifrado simétrico.
\end{lstlisting}

\subsection{Cifrado con contraseña}

\begin{lstlisting}[caption=Cifrado con contraseña y PBKDF2]
openssl enc -aes-256-cbc -pbkdf2 -in archivo.txt \
    -out archivo.bin -pass pass:"contraseña"

# Con información de derivación
openssl enc -aes-256-cbc -pbkdf2 -in archivo.txt \
    -out archivo.bin -pass pass:"contraseña" -p
\end{lstlisting}

\section{Resultados}

\subsection{Cifrado con diferentes algoritmos}

Se realizaron cifrados con los siguientes algoritmos:

\begin{table}[H]
    \centering
    \caption{Resultados del cifrado simétrico}
    \begin{tabularx}{\textwidth}{l|c|c|l}
        \toprule
        \textbf{Algoritmo} & \textbf{Modo} & \textbf{Tamaño} & \textbf{Descifrado} \\
        \midrule
        AES-256 & CBC & [Tamaño] bytes & $\checkmark$ OK \\
        AES-128 & CTR & [Tamaño] bytes & $\checkmark$ OK \\
        3DES & CBC & [Tamaño] bytes & $\checkmark$ OK \\
        3DES & OFB & [Tamaño] bytes & $\checkmark$ OK \\
        RC4 & - & [Tamaño] bytes & $\checkmark$ OK \\
        ChaCha20 & - & [Tamaño] bytes & $\checkmark$ OK \\
        AES-256 & GCM & [Tamaño] bytes & $\checkmark$ OK \\
        \bottomrule
    \end{tabularx}
\end{table}

\subsection{Análisis de overhead}

El tamaño de los archivos cifrados incluye:

\begin{itemize}[label=\textbullet]
    \item \textbf{Sal (Salt):} Primeros 8 bytes (formato: ``Salted\_\_'')
    \item \textbf{Contenido cifrado:} Datos cifrados
    \item \textbf{Padding (si aplica):} Relleno para completar bloques (en modos CBC)
\end{itemize}

Ecuación general:
\begin{equation}
\text{Tamaño\_cifrado} = 16 + \text{ceil}\left(\frac{\text{Tamaño\_original}}{T_{\text{bloque}}}\right) \times T_{\text{bloque}}
\end{equation}

Donde $T_{\text{bloque}}$ es el tamaño del bloque (128 bits para AES, 64 bits para 3DES).

\subsection{Cifrado con clave e IV explícitos}

Se demostró cómo descifrar utilizando la clave y el IV en lugar de la contraseña:

\begin{lstlisting}[caption=Cifrado con información de derivación]
# Obtener clave e IV derivados
openssl enc -aes-256-cbc -pbkdf2 -in archivo.txt \
    -out archivo.bin -pass pass:"contraseña" -p

# Eliminar los 16 bytes de sal
dd if=archivo.bin of=archivo_sin_sal.bin bs=1 skip=16

# Descifrar con clave e IV
openssl enc -aes-256-cbc -d -in archivo_sin_sal.bin \
    -out archivo_descifrado.txt -K <clave_hex> -iv <iv_hex>
\end{lstlisting}
